\documentclass[11pt,a4paper]{book}
\usepackage[utf8]{inputenc}
\usepackage[T1]{fontenc}
\usepackage{amsmath}
\usepackage{amsfonts}
\usepackage{amssymb}
\usepackage{graphicx}
\author{Korantin Lévêque}
\title{L'accord du participe passé en français standard écrit}
\begin{document}
\chapter{Introduction}
\section{Thème: Accès Lexical et Structure du Lexique}
\subsection{Accès lexical}
\subsection{Structure du Lexique}
\subsection{Problématique: En quoi le lexique à structure hautement complexe de type HPSG peut rendre un phénomène linguistique mieux décrit ?}
\subsection{Sujet: L'accord du participe passé en français standard écrit}
\paragraph{La vision syntaxique}
\paragraph{La vision Morphologique}
\paragraph{La vision morpho-syntaxique}
\paragraph{La vision lexicale}
\paragraph{Nouvelle vision lexicale}


\part{Développement}
\chapter{La vision syntaxique du problème}
\section{La grammaire générative}
\subsection{La morphologie distribuée}
\chapter{La vision morphologique}
\subsection{Théories inférentielles}
\subsection{Théories incrémentales}
\subsection{Théories réalisationnelles}
\chapter{Interface morphologie et syntaxe}
\subsection{La flexion}
\subsection{L'accord}
\chapter{HPSG: Apports et limites}
\section{Type}
\section{Trait}
\section{Structure}
\chapter{Vers une vision simplifiées des structures}
\section{Que cherche-t-on à décrire ?}
\subsection{Cherche-t-on à décrire ?}
\section{Génération/Reconnaissance}
\subsection{Reconnaissance initiale}
\subsection{introduction des types}
\section{Génération ensembliste}
\subsection{Introduction des notions d'intersection et d'union}
\section{Fréquence: Chaque type possède une fréquence relative}


\part{Conclusion}
\chapter{Révision intégrale de la distribution des tâches de la linguistique}
\chapter{La notion d'accord n'est plus un accord}
\chapter{La linguistique devient une interprétation de cette modélisation}

\end{document}