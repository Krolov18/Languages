\documentclass[a4paper,11pt]{article}
\usepackage[utf8]{inputenc}
\usepackage[french]{babel}
\usepackage[T1]{fontenc}


\begin{document}
La linguistique est un prolongement du locuteur. Ainsi la linguistique hérite du locuteur et ne peut pas être en contradiction avec celui-ci.
La linguistique se doit de décrire la compétence comme la performance du locuteur en observant ce dernier et en apprenant de celui-ci.
Si depuis plusieurs dizaines d'années, on se focalisait sur la composition syntagmatique de quoique ce soit en prenant le niveau syntagmatique comme primaire
dans le mécanisme, en contemplant les difficultés et en échouant à décrire un phénomène linguistique partant de ce principe, je change de vision et me focalise sur l'axe paradigmatique.
Le locuteur est alors présenté comme un utilisateur de l'analogie entre paradigmes. Sa performance, c'est-à-dire, la fréquence à laquelle, il utilise
ces analogies entre paradigmes lui fait acquérir une compétence linguistique. Cette compétence linguistique n'est pas du tout celle décrire par Chomsky et son programme minimaliste
La construction ne se fait pas tout le temps. L'apprentissage peut être mis en correspondance avec la construction (merge) de la syntaxe en ce sens qu'elle forge des connaissances
toutefois, elle n'est pas aussi simpliste qu'une simple fusion binaire. Selon la littérature et l'observation de la performance d'un locuteur, c'est à travers l'utilisation de celle-ci que la compétence se construit et
s'approfondit. Comme je le dis plus haut la linguistique est une extension du locuteur et elle ne doit pas aller contre le locuteur, on doit ressentir le même mécanisme à travers
la linguistique. 
On peut comparer le couple linguistique/locuteur avec mathématique/équation. L'équation mathématique nait d'une observation des faits et évolue au fur et à mesure que les faits s'empilent. 
On oublie souvent, surtout dans les matières formelles, qu'un algorithme aussi simple soit-il n'est pas mais est en devenir. E = MC2 n'est pas tombé du ciel, ce sont des années, des siècles de recherches
qui ont donné lieu à ce résultat. La linguistique est identique. Trouver la formule magique est une fin, ce qui nous intéresse, ce sont les moyens d' arriver.
Pour revenir à nos moutons, une phrase se compose d'une association paradigmatique. Lors de son apprentissage, l'enfant va d'abord écouter et observer avant de copier, répéter, se tromper.
Cet apprentissage est inductif la plupart du temps (j'irais même jusqu'à dire toujours). Si un graphe devait être représenté entre la performance et la compétence ce serait un graphe orienté qui irait
de la performance vers la compétence. Ce graphe va produire un système mémoriel double, les reflexes et la mémoire. Dans le fond ce système mémoriel n'est qu'un continuum, plus la fréquence est élevée plus on tend vers du réflexe
tandis que plus la fréquence est basse, plus on tend vers vers un niveau mémoriel plus complexe. On verra cela plus tard mais je pense que le lexique est inutile du point de vue du locuteur. Du moins la notion telle qu'elle
est dépeinte et utilisée aujourd'hui en linguistique doit être revue.
Le locuteur n'a conscience que de ce qu'il observe. Ainsi, ce qu'il n'a jamais observé n'existe pas pour lui. Selon moi en utilisant l'axe paradigmatique de manière fonctionnelle comme Stump 2001 nous le présente
il n'y a plus besoin du niveau syntaxique. Du moins, il faut le redéfinir. Si une séquence produite par un locuteur n'est que le fruit de l'intersection de plusieurs niveaux paradigmatiques, du coup, Ce qui différencie
la substitution de l'ajout (soit paradigmatique versus syntagmatique) c'est la compatibilité entre paradigmes ou devrais-je dire entre fonctions paradigmatiques. 
un paradigme se compose de plusieurs éléments: l'abstraction, une fonction paradigmatique. 
Ce modèle initialement ne cherche pas à générer mais à reproduire. Pour copier quelque chose, il faut avoir un modèle, pour parler une langue, il faut avoir DES modèles. En faisant des croisements d'analogie,
des groupes se forment, les groupes les plus fréquents seront les plus sensibles à avoir un lien avec la linguistique mais pas nécessairement. Il faut toujours envisager toutes les possibilités avec les pourcentages.
L'abstraction est une fonction de variabilisation d'une séquence. cette variabilisation est héritante. Dès qu'une variable apparait, cette variable aura tout au long de sa descendance la même valeur.
Le locuteur dont je parle depuis le départ reste un locuteur "idéal" en ce sens qu'il sera représenté comme un analogiseur parfait. Il sera capable de faire toute les analogies. Toutefois, ce locuteur idéal n'aura aucune préconception
comme celui de Chomsky. Le seul rapport avec l'inné qui pourrait éventuellement être rapporté est la notion d'analogie et de rapprochement/différenciation. Et encore, je ne suis même pas sûr
que ces notions soient tout à fait innées puisqu'un enfant ne différencie pas tout tout le temps et surtout l'enfant est beaucoup guidé par ses affects. Il ne fera ceci ou cela que si cela lui chante. Jouer avec les affects de l'enfant, c'est jouer avec
ses facultés d'appréhension et de compréhension.
Tandis qu'un paradigme abstrait des séquences, l'analogie met en relation des séquences, la syntaxe pourrait trouver sa place dans le fait de donner une fonction à un ensemble d'abstractions (paradigmes) qui ensemble forment une chaine.

\section{syntagmatique versus paradigmatique}
	On cherche depuis un certain temps, je ne saurais dire combien de temps maintenant, à construire des grammaire uniquement basées sur l'axe syntagmatique soit l'axe linéaire du temps. Toutes ces grammaires fonctionnent sur un seul et même principe
	la notion de règle. La grammaire traditionnelle cherchera le conventionnel et le bon usage de la langue tandis que des grammaires formelles chercheront les règles qui décriront ce que leurs auteurs voudront qu'elles décrivent. 
	Ces deux façons de penser sont selon moi, la même chose. Au lieu d'observer la pelote de laine telle qu'elle est, ces systèmes vont tenter de la démêler, instaurer des règles et tenter miraculeusement de retrouver la pelote d'origine. 
	Quand ils y arrivent, parce qu'il y arrivent par moment, ils instaurent alors des principes. Quand des "exceptions" émergent, ils vont parler de paramètres. 
	
	Chomsky et la grammaire générative s'est efforcé à isoler la connaissance. J'entends par isolement le principe binaire. La binarisation permet d'isoler un élément, simple, complexe, peu importe.
	Du coup, la logique minimaliste est de trouver la structure sous-jacente la plus simple et respectant une batterie de principes, règles, mouvements, transformations pour arriver une structure de surface pour alors subir
	des modifications phonologiques et être interprétées.
	
	Le but de cette section n'est pas de dénigrer ce type de grammaire, cette section me permet de montrer la différence entre la sphère syntagmatique et la sphère paradigmatique et de montrer la supériorité d'une grammaire paradigmatique à une syntagmatique.
	
	D'abord, un peu de définitions. Qu'est ce qu'un paradigme, formellement ? Je définierais un paradigme comme étant la partie gauche d'une règle de production $$(NT \vee T)^* T \vee NT (NT \vee T)^*$$ à la différence d'un symbole près, soit le NT qui est devenu un T.
	Tandis que T signifie terminal, et NT nonterminal comme dans les grammaires formelles, un paradigme n'est rien d'autre qu'une expression régulière. $$(je)(.*)(ai)(.*)$$ Ainsi ce type d'expression représente un paradigme.
	Ainsi un paradigme va se promenenr sur un continuum d'abstraction, allant d'un niveau concret soit $T_+$ jusqu'à un niveau complètement abstrait 
\section{grammaire paradigmatique}

\section{Structure de la grammaire}





\end{document}
